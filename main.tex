\let\negmedspace\undefined
\let\negthickspace\undefined
\RequirePackage{amsmath}
\documentclass[a4paper,13pt,twocolumn]{IEEEtran}
%
% \usepackage{setspace}
 \usepackage{gensymb}
 \usepackage{graphicx}
%\doublespacing
 \usepackage{polynom}
%\singlespacing
%\usepackage{silence}
%Disable all warnings issued by latex starting with "You have..."
%\usepackage{graphicx}
\usepackage{amssymb}
%\usepackage{relsize}

%\usepackage{amsthm}
%\interdisplaylinepenalty=2500
%\savesymbol{iint}
%\usepackage{txfonts}
%\restoresymbol{TXF}{iint}
%\usepackage{wasysym}
\usepackage{amsthm}
%\usepackage{pifont}
%\usepackage{iithtlc}
% \usepackage{mathrsfs}
% \usepackage{txfonts}
 \usepackage{stfloats}
% \usepackage{steinmetz}
 \usepackage{bm}
% \usepackage{cite}
% \usepackage{cases}
% \usepackage{subfig}
%\usepackage{xtab}
\usepackage{longtable}
%\usepackage{multirow}
%\usepackage{algorithm}
%\usepackage{algpseudocode}
\usepackage{enumitem}
 \usepackage{mathtools}
 \usepackage{tikz}
% \usepackage{circuitikz}
% \usepackage{verbatim}
%\usepackage{tfrupee}
\usepackage[breaklinks=true]{hyperref}
%\usepackage{stmaryrd}
%\usepackage{tkz-euclide} % loads  TikZ and tkz-base
%\usetkzobj{all}
\usepackage{listings}
    \usepackage{color}                                            %%
    \usepackage{array}                                            %%
    \usepackage{longtable}                                        %%
    \usepackage{calc}                                             %%
    \usepackage{multirow}                                         %%
    \usepackage{hhline}                                           %%
    \usepackage{ifthen}                                           %%
  %optionally (for landscape tables embedded in another document): %%
    \usepackage{lscape}     
% \usepackage{multicol}
% \usepackage{chngcntr}
%\usepackage{enumerate}
    \usepackage{amsmath}
%\usepackage{wasysym}
%\newcounter{MYtempeqncnt}

\DeclareMathOperator*{\Res}{Res}
\DeclareMathOperator*{\equals}{=}
%\renewcommand{\baselinestretch}{2}


% correct bad hyphenation here
\hyphenation{op-tical net-works semi-conduc-tor}
                                %%

\lstset{
%language=C,
frame=single, 
breaklines=true,
columns=fullflexible
}

\begin{document}

\newtheorem{theorem}{Theorem}[section]
\newtheorem{problem}{Problem}
\newtheorem{proposition}{Proposition}[section]
\newtheorem{lemma}{Lemma}[section]
\newtheorem{corollary}[theorem]{Corollary}
\newtheorem{example}{Example}[section]
\newtheorem{definition}[problem]{Definition}
%\newtheorem{thm}{Theorem}[section] 
%\newtheorem{defn}[thm]{Definition}
%\newtheorem{algorithm}{Algorithm}[section]
%\newtheorem{cor}{Corollary}
\newcommand{\BEQA}{\begin{eqnarray}}
\newcommand{\EEQA}{\end{eqnarray}}
\newcommand{\define}{\stackrel{\triangle}{=}}
\newcommand*\circled[1]{\tikz[baseline=(char.base)]{
    \node[shape=circle,draw,inner sep=2pt] (char) {#1};}}
\bibliographystyle{IEEEtran}
%\bibliographystyle{ieeetr}
\providecommand{\mbf}{\mathbf}
\providecommand{\pr}[1]{\ensuremath{\Pr\left(#1\right)}}
\providecommand{\qfunc}[1]{\ensuremath{Q\left(#1\right)}}
\providecommand{\sbrak}[1]{\ensuremath{{}\left[#1\right]}}
\providecommand{\lsbrak}[1]{\ensuremath{{}\left[#1\right.}}
\providecommand{\rsbrak}[1]{\ensuremath{{}\left.#1\right]}}
\providecommand{\brak}[1]{\ensuremath{\left(#1\right)}}
\providecommand{\lbrak}[1]{\ensuremath{\left(#1\right.}}
\providecommand{\rbrak}[1]{\ensuremath{\left.#1\right)}}
\providecommand{\cbrak}[1]{\ensuremath{\left\{#1\right\}}}
\providecommand{\lcbrak}[1]{\ensuremath{\left\{#1\right.}}
\providecommand{\rcbrak}[1]{\ensuremath{\left.#1\right\}}}
\theoremstyle{remark}
\newtheorem{rem}{Remark}
\newcommand{\sgn}{\mathop{\mathrm{sgn}}}
%\providecommand{\hilbert}{\overset{\mathcal{H}}{ \rightleftharpoons}}
\providecommand{\system}{\overset{\mathcal{H}}{ \longleftrightarrow}}
	%\newcommand{\solution}[2]{\textbf{Solution:}{#1}}
\newcommand{\solution}{\noindent \textbf{Solution: }}
\newcommand{\cosec}{\,\text{cosec}\,}
\providecommand{\dec}[2]{\ensuremath{\overset{#1}{\underset{#2}{\gtrless}}}}
\newcommand{\myvec}[1]{\ensuremath{\begin{pmatrix}#1\end{pmatrix}}}
\newcommand{\mydet}[1]{\ensuremath{\begin{vmatrix}#1\end{vmatrix}}}
\newcommand{\taninv}{\tan^{-1}}
\renewcommand{\figurename}{Fig.}
\makeatletter
\@addtoreset{figure}{problem}
\makeatother
\let\StandardTheFigure\thefigure
\let\vec\mathbf
%\renewcommand{\thefigure}{\theproblem.\arabic{figure}}
%\setlist[enumerate,1]{before=\renewcommand\theequation{\theenumi.\arabic{equation}}
%\counterwithin{equation}{enumi}
%\renewcommand{\theequation}{\arabic{subsection}.\arabic{equation}}
\def\putbox#1#2#3{\makebox[0in][l]{\makebox[#1][l]{}\raisebox{\baselineskip}[0in][0in]{\raisebox{#2}[0in][0in]{#3}}}}
     \def\rightbox#1{\makebox[0in][r]{#1}}
     \def\centbox#1{\makebox[0in]{#1}}
     \def\topbox#1{\raisebox{-\baselineskip}[0in][0in]{#1}}
     \def\midbox#1{\raisebox{-0.5\baselineskip}[0in][0in]{#1}}
 
\vspace{3cm}
\title{ASSIGNMENT-2 : ICSE-2019, 12th GRADE} 
\author{Sathvika mari\\AI21BTECH11020}
\maketitle
\bigskip

\textbf{PROBLEM 5-B :}
(b).Verify the Lagrange’s mean value theorem for the function:
f(x) = x + 1/x in the interval [1, 3].\\

\textbf{SOLUTION :}\\
f(x) = x + $\frac{1}{x}$ be in closed interval $1\leq x\leq 3$, i.e [1,3]\\
f$^\prime$(x) = 1 - $\frac{1}{x^2}$ is existing in the open interval $1 < x < 3 $, i.e (1,3).\\
Since, f(x) is a polynomial function, therefore, it is continuous and derivable in (1, 3).\\
The conditions of lagrange's mean value theorem are satisfied.\\
f(1) = 1+1 = 2 f(3) = 3 + $\frac{1}{3}$ = $\frac{10}{3}$.\\
To verify further, need to show that there exists a 'c' $\in$ (1,3) such that,

\begin{align}
    \implies f^\prime(c) = \frac{f(b)-f(a)}{b-a}
    \label{eq:eq1}
\end{align}
\begin{align}
    \implies 1 - \frac{1}{x^2} = \frac{\frac{10}{3}-2}{3-1}
    \label{eq:eq2}
\end{align}
\begin{align}
    \implies \frac{4}{3} \times \frac{1}{2} = \frac{2}{3}
    \label{eq:eq3}
\end{align}
\begin{align}
    \implies 1 - \frac{1}{x^2} = \frac{2}{3}
    \label{eq:eq4}
\end{align}
\begin{align}
    \implies \frac{1}{x^2} = 1 - \frac{2}{3} = \frac{1}{3}
    \label{eq:eq5}
\end{align}
\begin{align}
    \implies x^2 = 3
    \label{eq:eq6}
\end{align}

\begin{align}
    \implies \boxed{x = \pm \sqrt{3}}
\end{align}


\centering{
\therefore $+\sqrt{3}$ (or) +1.732 lies in the open interval (1,3) or c $in$ (1,3)}\\
Hence, Mean Value theorem for the given function is verified in the given interval.



\end{document}